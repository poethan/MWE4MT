\documentclass[11pt]{extarticle}

% fontspec allows to import fonts, ucharclasses select the font
\usepackage{fontspec}
\usepackage{ucharclasses}

% We import the fonts we need
\newfontfamily\SourceHanSerif{Source Han Serif}
\newfontfamily\HamazonoMinA{Hanazono Mincho A}
\newfontfamily\HamazonoMinB{Hanazono Mincho B}
\newfontfamily\HamazonoMinC{Hanazono Mincho C}
\newfontfamily\IPAexMincho{IPAexMincho}


% default font, computer modern
\setDefaultTransitions{\fontfamily{lmodern}\selectfont}{}

% for each unicode class we setup the font
\setTransitionTo{CJKUnifiedIdeographs}{\SourceHanSerif}
\setTransitionTo{CJKUnifiedIdeographsExtensionA}{\HamazonoMinA}
\setTransitionTo{CJKUnifiedIdeographsExtensionB}{\HamazonoMinB}
\setTransitionTo{CJKUnifiedIdeographsExtensionC}{\HamazonoMinC}
\setTransitionTo{CJKUnifiedIdeographsExtensionE}{\HamazonoMinC}
\setTransitionTo{IdeographicDescriptionCharacters}{\IPAexMincho}
\setTransitionTo{EnclosedAlphanumerics}{\IPAexMincho}


% To detect the class of a character a possibility is:
% . go to duckduckgo.com
% . search for "unicode 鬯" (or the character you want, without quotes)
% . before the first result there will be a line like:
%   鬯 U+9B2F CJK UNIFIED IDEOGRAPH-9B2F, decimal: 39727, HTML: &#39727;, UTF-8: 0xE9 0xAC 0xAF, script: Han, block: CJK Unified Ideographs
% . Look up for the ucharclasses block in the documentation
%   http://ftp.heanet.ie/pub/ctan.org/tex/macros/xetex/latex/ucharclasses/ucharclasses.pdf


% To detect what font has a character one possiblity is:
% . open libreoffice writer
% . type the character
% . if the system shows it properly, export the file to pdf
% . use pdffonts to see what fonts are inside the pdf
% . the name will be mangled, but hopefully detectable in the fc-list output


\usepackage{url}
\begin{document}

\section{Example}

For the Chinese Characters decompose, we used the IDS data for CJK Unified Ideographs\footnote{\url{https://github.com/cjkvi/cjkvi-ids}}; in the repo the \texttt{ids.txt} file contains a mapping between almost 90 thousands Chinese Japanese or Korean characters and radicals and their decomposition.  For the characters with multiple decompositions we used the one from mainland China.  For example 乢 maps to ⿰山乚; complex characters can often decomposed further.  For example 鬱 maps to ⿳⿲木缶木冖⿰鬯彡, its component 鬯 maps to ⿱𠚍匕, and again 𠚍 maps to ⿶凵𠂭.

The circled integer (①--⑳) are used to denote the number strokes where the representing character is not known or not available to Unicode. For example 𬋢 expands to ⿱⿰禾⑲火.


\end{document}

